\documentclass{article}\usepackage[]{graphicx}\usepackage[]{xcolor}
% maxwidth is the original width if it is less than linewidth
% otherwise use linewidth (to make sure the graphics do not exceed the margin)
\makeatletter
\def\maxwidth{ %
  \ifdim\Gin@nat@width>\linewidth
    \linewidth
  \else
    \Gin@nat@width
  \fi
}
\makeatother

\definecolor{fgcolor}{rgb}{0.345, 0.345, 0.345}
\newcommand{\hlnum}[1]{\textcolor[rgb]{0.686,0.059,0.569}{#1}}%
\newcommand{\hlstr}[1]{\textcolor[rgb]{0.192,0.494,0.8}{#1}}%
\newcommand{\hlcom}[1]{\textcolor[rgb]{0.678,0.584,0.686}{\textit{#1}}}%
\newcommand{\hlopt}[1]{\textcolor[rgb]{0,0,0}{#1}}%
\newcommand{\hlstd}[1]{\textcolor[rgb]{0.345,0.345,0.345}{#1}}%
\newcommand{\hlkwa}[1]{\textcolor[rgb]{0.161,0.373,0.58}{\textbf{#1}}}%
\newcommand{\hlkwb}[1]{\textcolor[rgb]{0.69,0.353,0.396}{#1}}%
\newcommand{\hlkwc}[1]{\textcolor[rgb]{0.333,0.667,0.333}{#1}}%
\newcommand{\hlkwd}[1]{\textcolor[rgb]{0.737,0.353,0.396}{\textbf{#1}}}%
\let\hlipl\hlkwb

\usepackage{framed}
\makeatletter
\newenvironment{kframe}{%
 \def\at@end@of@kframe{}%
 \ifinner\ifhmode%
  \def\at@end@of@kframe{\end{minipage}}%
  \begin{minipage}{\columnwidth}%
 \fi\fi%
 \def\FrameCommand##1{\hskip\@totalleftmargin \hskip-\fboxsep
 \colorbox{shadecolor}{##1}\hskip-\fboxsep
     % There is no \\@totalrightmargin, so:
     \hskip-\linewidth \hskip-\@totalleftmargin \hskip\columnwidth}%
 \MakeFramed {\advance\hsize-\width
   \@totalleftmargin\z@ \linewidth\hsize
   \@setminipage}}%
 {\par\unskip\endMakeFramed%
 \at@end@of@kframe}
\makeatother

\definecolor{shadecolor}{rgb}{.97, .97, .97}
\definecolor{messagecolor}{rgb}{0, 0, 0}
\definecolor{warningcolor}{rgb}{1, 0, 1}
\definecolor{errorcolor}{rgb}{1, 0, 0}
\newenvironment{knitrout}{}{} % an empty environment to be redefined in TeX

\usepackage{alltt}
\usepackage[sc]{mathpazo}
\renewcommand{\sfdefault}{lmss}
\renewcommand{\ttdefault}{lmtt}
\usepackage[T1]{fontenc}
\usepackage{geometry}
\geometry{verbose,tmargin=2.5cm,bmargin=2.5cm,lmargin=2.5cm,rmargin=2.5cm}
\setcounter{secnumdepth}{2}
\setcounter{tocdepth}{2}
\usepackage[unicode=true,pdfusetitle,
 bookmarks=true,bookmarksnumbered=true,bookmarksopen=true,bookmarksopenlevel=2,
 breaklinks=false,pdfborder={0 0 1},backref=false,colorlinks=false]
 {hyperref}
\hypersetup{
 pdfstartview={XYZ null null 1}}

\makeatletter
%%%%%%%%%%%%%%%%%%%%%%%%%%%%%% User specified LaTeX commands.
\renewcommand{\textfraction}{0.05}
\renewcommand{\topfraction}{0.8}
\renewcommand{\bottomfraction}{0.8}
\renewcommand{\floatpagefraction}{0.75}

\makeatother
\IfFileExists{upquote.sty}{\usepackage{upquote}}{}
\begin{document}



\title{\title{\title{}}}



\maketitle
The results below are generated from an R script.

\begin{knitrout}
\definecolor{shadecolor}{rgb}{0.969, 0.969, 0.969}\color{fgcolor}\begin{kframe}
\begin{alltt}
\hlcom{# Assignment: ASSIGNMENT 0}
\hlcom{# Name: Tang, Xin}
\hlcom{# Date: 2023-06-11}

\hlcom{# Basics}

\hlcom{## Add 8 and 5}
\hlnum{8}\hlopt{+}\hlnum{5}
\end{alltt}
\begin{verbatim}
## [1] 13
\end{verbatim}
\begin{alltt}
\hlcom{## Subtract 6 from 22}
\hlnum{22}\hlopt{-}\hlnum{6}
\end{alltt}
\begin{verbatim}
## [1] 16
\end{verbatim}
\begin{alltt}
\hlcom{## Multiply 6 by 7}
\hlnum{6}\hlopt{*}\hlnum{7}
\end{alltt}
\begin{verbatim}
## [1] 42
\end{verbatim}
\begin{alltt}
\hlcom{## Add 4 to 6 and divide the result by 2}
\hlstd{(}\hlnum{4}\hlopt{+}\hlnum{6}\hlstd{)}\hlopt{/}\hlnum{2}
\end{alltt}
\begin{verbatim}
## [1] 5
\end{verbatim}
\begin{alltt}
\hlcom{## Compute 5 modulo 2}
\hlnum{5}\hlopt\hlnum{2}
\end{alltt}
\begin{verbatim}
## [1] 1
\end{verbatim}
\begin{alltt}
\hlcom{## Assign the value 82 to the variable x}
\hlcom{## Print x}
\hlstd{x} \hlkwb{<-} \hlnum{82}
\hlstd{x}
\end{alltt}
\begin{verbatim}
## [1] 82
\end{verbatim}
\begin{alltt}
\hlcom{## Assign the value 41 to the variable y}
\hlcom{## Print y}
\hlstd{y} \hlkwb{<-} \hlnum{41}
\hlstd{y}
\end{alltt}
\begin{verbatim}
## [1] 41
\end{verbatim}
\begin{alltt}
\hlcom{## Assign the output of x + y to the variable z}
\hlcom{## Print z}
\hlstd{z} \hlkwb{<-} \hlstd{x}\hlopt{+}\hlstd{y}
\hlkwd{print}\hlstd{(z)}
\end{alltt}
\begin{verbatim}
## [1] 123
\end{verbatim}
\begin{alltt}
\hlcom{## Assign the string value "DSC520" to the variable class_name}
\hlcom{## Print the value of class_name}
\hlstd{class_name} \hlkwb{<-} \hlstr{"DSC520"}
\hlkwd{print}\hlstd{(class_name)}
\end{alltt}
\begin{verbatim}
## [1] "DSC520"
\end{verbatim}
\begin{alltt}
\hlcom{## Assign the string value of TRUE to the variable is_good}
\hlcom{## Print the value of is_good}
\hlstd{is_good} \hlkwb{<-} \hlnum{TRUE}
\hlkwd{print}\hlstd{(is_good)}
\end{alltt}
\begin{verbatim}
## [1] TRUE
\end{verbatim}
\begin{alltt}
\hlcom{## Check the class of the variable is_good using the `class()` function}
\hlkwd{class}\hlstd{(is_good)}
\end{alltt}
\begin{verbatim}
## [1] "logical"
\end{verbatim}
\begin{alltt}
\hlcom{## Check the class of the variable z using the `class()` function}
\hlkwd{class}\hlstd{(z)}
\end{alltt}
\begin{verbatim}
## [1] "numeric"
\end{verbatim}
\begin{alltt}
\hlcom{## Check the class of the variable class_name using the class() function}
\hlkwd{class}\hlstd{(class_name)}
\end{alltt}
\begin{verbatim}
## [1] "character"
\end{verbatim}
\end{kframe}
\end{knitrout}

The R session information (including the OS info, R version and all
packages used):

\begin{knitrout}
\definecolor{shadecolor}{rgb}{0.969, 0.969, 0.969}\color{fgcolor}\begin{kframe}
\begin{alltt}
\hlkwd{sessionInfo}\hlstd{()}
\end{alltt}
\begin{verbatim}
## R version 4.3.0 (2023-04-21 ucrt)
## Platform: x86_64-w64-mingw32/x64 (64-bit)
## Running under: Windows 10 x64 (build 19045)
## 
## Matrix products: default
## 
## 
## locale:
## [1] LC_COLLATE=English_United States.utf8  LC_CTYPE=English_United States.utf8   
## [3] LC_MONETARY=English_United States.utf8 LC_NUMERIC=C                          
## [5] LC_TIME=English_United States.utf8    
## 
## time zone: America/Chicago
## tzcode source: internal
## 
## attached base packages:
## [1] stats     graphics  grDevices utils     datasets  methods   base     
## 
## other attached packages:
## [1] tinytex_0.45 knitr_1.43  
## 
## loaded via a namespace (and not attached):
## [1] compiler_4.3.0 tools_4.3.0    highr_0.10     xfun_0.39      evaluate_0.21
\end{verbatim}
\begin{alltt}
\hlkwd{Sys.time}\hlstd{()}
\end{alltt}
\begin{verbatim}
## [1] "2023-06-12 22:48:49 CDT"
\end{verbatim}
\end{kframe}
\end{knitrout}


\end{document}
